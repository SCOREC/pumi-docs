%%%%%%%%%%%%%%%%%%%%%%%%%%%%%%%%%%%%%%%%%%%%%%%%%%%%%%%%%%%%%%%%%%% 
%                       rpithes-short.tex                         %
%         Template for a short thesis all in one file             %
%        (titlepage info below assumes masters degree}            %
%  Just run latex (or pdflatex) on this file to see how it looks  %
%      Be sure to run twice to get correct TOC and citations      %
%%%%%%%%%%%%%%%%%%%%%%%%%%%%%%%%%%%%%%%%%%%%%%%%%%%%%%%%%%%%%%%%%%% 
%
%  To produce the abstract title page followed by the abstract,
%  see the template file, "abstitle-mas.tex"
%
%%%%%%%%%%%%%%%%%%%%%%%%%%%%%%%%%%%%%%%%%%%%%%%%%%%%%%%%%%%%%%%%%%%

\documentclass{thesis}
\usepackage[english]{babel}
\usepackage[utf8x]{inputenc}
\usepackage{amsmath}
\usepackage{graphicx}   % if you want to include graphics files
\usepackage{algorithm}
\usepackage{algorithmicx}
\usepackage{algpseudocode}
\usepackage{url}
\usepackage[font=small,labelfont=bf]{caption}
\renewcommand{\algorithmicforall}{\textbf{for each}}
\let\ForEach\ForAll
% Use the first command below if you want captions over 1 line indented.
% A side effect of this is to remove the use of bold for captions. 
% To restore bold, also include the second line below.
%\usepackage[hang]{caption}     % to indent subsequent lines of captions
%\renewcommand{\captionfont}{\bfseries} % only needed with caption package;
                                        %   otherwise bold is default)
                                        
%%%%%%%%%%%%%%%%%%%%  supply titlepage info  %%%%%%%%%%%%%%%%%%%%%
\thesistitle{\bf Diffusive Load Balancing\\of Shape and Neighbor Counts}        
\author{Gerrett Diamond}
\degree{Bachelors of Science}
\department{Mathematics} % provide your area of study here; e.g.,
%  "Mechanical Engineering", "Nuclear Engineering", "Physics", etc.
\thadviser{Mark Shephard}
%\cothadviser{First co-adviser} %if needed
%\cocothadviser{Second co-adviser} % if needed
%  For a masters project use \projadviser instead of \thadviser, 
%  and \coprojadviser and \cocoprojadviser if needed. 
\submitdate{March 2015\\(For Graduation May 2015)}        
%\copyrightyear{1685}  % if date omitted, current year is used. 
%%%%%%%%%%%%%%%%%%%%%   end titlepage info  %%%%%%%%%%%%%%%%%%%%%%
      
\begin{document} 
\titlepage             % Print titlepage   
\tableofcontents       % required 
%\listoftables          % required if there are tables
%\listoffigures         % required if there are figures

\specialhead{ABSTRACT}
Massively parallel mesh simulations require the mesh to be distributed
throughout many processes (parts) and be balanced in terms of computation and
communication. 
Many algorithms exist that target having an equal number mesh entities of a
specified dimension on each part. 
The most powerful of these, in terms of partition quality and run time are the
multi-level (hyper)graph-based methods and the recursive geometric sectioning
methods. 
Multi-level (hyper)graph methods describe the mesh in terms of graph nodes and
(hyper)edges and then use a v-cycle of coarsening followed by un-coarsening to
divide the graph into subgraphs while recursive sectioning geometric methods use
a coordinate system such as the Cartesian location or centroids to recursively
section the mesh along coordinate or inertial axis. 
The faster geometric methods tend to create parts that have large inter-part
surface area, and many neighbors relative to the (hyper)graph based methods
which causes an increase in communication.
(Hyper)Graph-based methods can also suffer from similar issues as the number of
parts gets very large and the number of elements per part drops to several
hundreds. 
An alternative method that uses mesh data directly, ParMA, performs load 
balancing without the need to construct a graph. 
With access to all the mesh, partition, and model data, including topological 
and geometric coordinates, partitioning methods analogous to those used by 
(hyper)graph and geometric based procedures can be defined.
Towards this, ParMA partitioning methods were extended to improve the 
overall shape of a part in order to reduce the number of neighbors each part 
has and increase small part boundaries and thus reduce application 
communication times.

\newpage

\begingroup
\let\clearpage\relax

\chapter{INTRODUCTION}

%Descrube an unstructured mesh and its uses in the scientific world
In many scientific applications, mathematical models are used to develop 
simulations of real world scenarios. These models often are discretized over
either two dimensional or three dimensional unstructured mesh of vertices, 
edges, 
faces, and regions (for three dimensional meshes).  Complexities in the model 
and the domain can easily require meshes with billions of elements.
To run a simulation of this size it is necessary to distribute the elements 
across hundreds of thousands of processors where each processor gets a portion 
of the elements, a mesh 'part'.

To ensure the effectiveness of distributing the mesh among parts it is 
important to balance the loads evenly throughout the parts. This load is 
expressed as the combination of the computational costs and the communication 
costs which depend on the application. Computational costs include the 
application's work that has to be done on the local elements of the part. The 
communication costs involve sending and receiving information between parts, 
often defined by mesh entity adjacencies across part boundaries. Balancing 
both of these based on the application leads to the least overhead of the 
application.

%Massage this a bit
When communication costs are measured by, or a direct result of, the number 
of entities cut by part boundaries the costs can be lowered by (1) reducing 
the number of entities cut by the boundary and (2) reducing the number of 
parts, processes, a part borders. Objective 1 is a typical objective of most 
partitioning tools which is to reduce the cut size. The latter objective, 
assuming equal cut size, reflects the potentially increased cost of sending 
several small messages to many endpoints versus sending large 
messages to fewer endpoints \cite{MPIpresent,MPImessage}. 
%should I mention something about how we target the second point.

Typical parallel mesh load balancing algorithms target distributing the 
computational load of an application evenly across all parts. Even with the 
computational load of each part balanced, a part with many neighbors can have 
increased communication overheads. In order to fix such boundary problems a 
balancing algorithm must take into account the overall shape of the parts. 
More precisely the algorithm must try to remove part boundaries cutting only a 
small number of mesh vertices.

We present  a method that given a mesh will 
increase the length of small part boundaries. Our work 
sends a small number of elements bounded by the ends, in 2D the endpoints of 
the line formed by the part boundary, in 3D the segments bounding the part 
boundary surface, of the targeted part boundary to incrementally increase its 
size. The format of the paper is as follows. Section 2 
discusses common approaches to mesh load balancing and there use in shape 
correction. Section 3 describes our algorithm to fix part shape. Section 4 
shows results from our testing. Section 5 and 6 list our plans to improve the 
method. 

Some terminology used in the paper is defined as follows:

{\bf Mesh Entity } - A piece of the mesh representation of order $d$ where $d \in [0,3]$ which refers to vertex, edge, face, region in order.

{\bf Element} - A mesh entity that is a face for a two dimensional mesh or a region for a three dimensional mesh.

{\bf Imbalance of order $d$} - the maximum number of entities of order $d$ on any part divided by the average across all parts

{\bf Boundary length} - the number of vertices on the boundary between two parts.

{\bf Small Boundary} - a partition model edge/face between two parts that has a boundary length less than a given percentage of the average boundary length.


\chapter{Related Works}
A number of distributed methods to balance unstructured meshes have been 
explored in the past \cite{multidiffuse,surveygraph}. Here we overview the 
three most commonly used methods: graph/hypergraph, diffusive, and multilevel. 
Graph and Hypergraph methods balance the mesh based on a graph structure built 
on top of the mesh. Diffusive methods migrate elements from parts with high 
imbalance to neighboring parts with lower imbalance. Multilevel methods use 
a combination of load balancing methods to balance a mesh starting at a 
coarser level and maintaining the balance as it is refined.

\section{(Hyper)graph methods}
Graph and Hypergraph load balancing methods use an abstraction of the mesh in 
order to balance the work load between processes \cite{surveygraph}. The 
first step to using 
hypergraph/graph methods is creating this abstraction. Zhou et al. define that 
the graph nodes are built off of a specific mesh entity typically either faces 
for 2D or regions for 3D \cite{zhougraph}. The edges are then built based on the
dependencies of the nodes. In their work they found that using only a subset of 
these edges is sufficient for the same quality of partition. For their work with
regions as nodes they found using the face adjancencies gave just as good a 
partition in much less time. The authors also define two different types of 
edges: Interedges that connect graph nodes on different processors and 
intraedges that connect graph nodes on the same processor. Zhou describes the 
hypergraph as an extension of the graph where hypergraph nodes are the same as 
graph nodes, but hypergraph edges or hyperedges represent dependencies between 
more than one mesh entity. Thus hyperedges can correlate each entity with all 
of its face adjancencies. The overall goal of hypergraphs and hyperedges is to 
partition the graph while minimizing the number of parts the hyperedges extend 
over.

Buluc et al. \cite{surveygraph} give an overview of graph methods as used in 
load balancing 
for both graphs and hypergraphs. They also discuss the challenges and 
NP-hardness of many of the algorithms for graph balancing. Two metrics that 
are used to model the overall shape for graph methods are the edge cut and 
communication volume. The authors show that communication volume is a better 
measurement which leads to hypergraph methods which better target the 
communication volume because hyperedges represent the number of different 
neighboring parts to each node in the graph. Devine et al. discuss methods 
to use hypergraphs for parallel scientific computing \cite{hypergraph}. They
use hypergraphs to accurately represent communication volume and as a result
part neighbors. The hypergraph method uses hyperedges or nets which connects 
several nodes in the graph that may be on different parts. Therefore 
balancing a hypergraph naturally puts an effort towards balancing the overall
shape and neighbors of parts. 

%Maybe add talking about how local cannot fix bad partitions
There are some draw backs to graph and hypergraph methods as discussed 
by Zhou et al. \cite{zhougraph}. First, since the graph targets 
entities of one dimension, the entities of other dimensions can become 
unbalanced. Also graph/hypergraph methods do not scale well when working with 
more than 
100,000 processors \cite{zhougraph}. The authors explain that using both 
local partitioning where each part uses its own mesh data to partition  
and global partitioning where each part uses the entire mesh data can result 
in good partitions that can scale out to
large part counts. Also with large part counts some parts become much 
heavier than others which limits the scalability of these methods.

\section{Diffusive load balancing}
Willebeek-LeMair and Reeves describe diffusive load balancing techniques with 
four steps \cite{loadbalance}. The first is evaluating the processor load, then 
determining the profitability of the load balancing, followed by a task 
migration strategy, and finally a task selection strategy. They review 
five different strategies and how they perform on the second and third task. 
The first method reviewed was the Sender Initiated Diffusion method (SID). 
This method restricts parts to only communicate with their neighbors. The 
parts with a large imbalance, refered to as heavy parts, are the migration 
selectors or sender parts. The second technique is 
similar to the first and called Receiver Initiated Diffusion (RID). Once again 
parts can only communicate with their neighbors, but instead of the heavy parts
selecting migration targets, the parts with lower imbalance select heavier 
neighbors to receive
load from. The third method is the Gradient Method (GM) where light parts send 
messages to all of the parts asking to receive load. Heavy parts then choose 
the closest light part and migrate elements to neighboring parts in that 
direction. The Hierarchical Balancing Method (HBM) breaks up load balancing 
into a 
hierarchy of parts and performs load balancing at a low level and then continues
to balance while moving up a hierarchy. The final method is the Dimension 
Exchange Method (DEM). Similar to the HBM, the DEM approach balances at a low 
level and works up, but the DEM balances synchronously in all dimensions one at 
a time.

Zhou et al. developed a diffusive algorithm to balance the heavy parts produced 
by graph/hypergraph methods \cite{zhougraph}. They target balancing the mesh vertices for each
part of the partition by offloading certain entities from heavy parts to 
lighter parts. Their method, the local iterative interpart boundary 
modification algorithm (LIIPBMod), identifies mesh vertices on the heavy parts
that are bounded by parts with a small number of elements. By doing this, LIIPBMod 
balances the vertex partitioning while still maintaining the element balance
produced from the graph/hypergraph partition. This process when repeated 
%not sure of Mark's suggestion
results in a partition that has vertex balance with a minor decrease in the 
good element balance. The results of their method shows a dramatic improvement
of vertex imbalance sometimes reducing from over 20\% to around 5\%. The improvement of the vertex imbalance has a cost of increasing the element imbalance. For their tests the element imbalance increased by 1\% to 3\% starting at an imbalance of 2\%. %check this with cameron
The decrease in total solution time due to the improved partition quality is much larger than the time required to improve the partition.

\section{Multilevel methods}
Buluc et al. describe the multilevel methods as the most successful approach to 
graph partitioning \cite{surveygraph}. They describe multilevel methods 
generally as a three step process. The first step is to coarsen the mesh 
until some threshold in order to reduce complexity. The second is to create 
an initial partition of this fully coarsened mesh that can be done much 
faster than the entire mesh. Next the mesh is refined in the opposite direction
of the coarsening. At each step the mesh is progressively refined in order to
keep the balance across the parts. This design is called a v-cycle. The 
generality of the design allows most algorithms to be altered to work in a 
multilevel form. 

Meyerhenke et al. port a diffusive method previously used for 
shape optimization into a multilevel algorithm \cite{multidiffuse}. They 
employ a 
previously used distributed load balancing algorithm, Bubble-FOS/C (First order
diffusion scheme with constant drain) for load balancing in their multilevel
method. This Bubble method is an iterative scheme based on Lloyd's k-means 
algorithm. This method involves solving linear systems iteratively which causes 
a high run time. Thus the authors introduce a second balancing algorithm, 
TruncCons (truncated diffusion consolidations), into the v-cycle. This 
algorithm is faster than Bubble for large mesh sizes and is used for small
diffusive improvements. They use the multilevel structure in order to use 
the expensive but more effective Bubble-FOS/C when at the the lower levels of 
the v-cycle because the parts are smaler and then use TruncCons to balance the 
larger parts of the v-cycle. 
This method was shown to be much faster than the original method 
of just the Bubble-FOS/C and also produces a better quality of load balancing. 
They compared this new method to the popular methods KMetis and JOSTLE and 
found their method to produce higher quality partitions but still having a
much higher run time. They expect reasonable scaling for the algorithm but 
results show the algorithm taking minutes rather than seconds like other 
methods.

The diffusive method for load balancing is heavily used throughout our 
algorithm as our method iteratively migrates elements by direcly using the 
mesh entities. Our method creates a graph like structure of the parts in 
order to represent part adjacencies similar to graph and hypergraph approaches 
to load balancing. Although we do not implement a multilevel approach, our 
algorithm does operate on multiple levels of the mesh and partition 
granularity like the mesh, partition model and entity topology.

\chapter{Shape Improvement Algorithm}

The approach we take to fixing small patition faces builds off a parallel 
dynamic partitioning framework called ParMA \cite{parma}. Thus our algorithm, 
Gap, defines a series of steps defined in the ParMA load balancing method. 
Currently Gap targets small partition model faces and migrate elements to 
increase the size of these boundaries. There is also the possiblity to remove 
small partition model faces, but we have not implemented this in Gap. Here we 
describe the base framework of ParMA and how it is extended to create Gap.

\section{ParMA}

ParMA load balancing methods follow a five step plan. These five steps 
are the sides determination, weight assignment, target phase, selection
phase, and the stepper phase. The sides determination is when each part 
determines which parts are its neighbors. These sides are defined for some
 dimension $d$ as a part that shares a mesh entity of order $d$. After sides are 
determined, the weight assignment phase collects the weight (for a certain 
mesh entity order (vertex, edge, face, or region)) of the neighbor parts 
defined by the sides. The 
target phase defines which parts will be sending and the parts each one 
will be sending to. The target phase also defines an amount of weight to 
be transferred from each sending part to each of its targets. The selection 
phase determines which elements of the mesh will be sent to the targets 
until the given amount of weight is satisfied. The final stepper phase is 
to migrate the elements between each part and complete the load balancing 
step. These steps are repeated until a stopping criteria is satisfied 
during the stepper phase. 

Each diffusive load balancing algorithm derived from ParMA implements these 
steps to target the criteria of the algorithm \cite{parma}. The most important 
steps of the ParMA framework is the target phase and selection phase which 
defines the problem specific actions that will be taken to balance the parts. 
The stopping criteria is typically to go below some balancing threshold of 
a given mesh entity order(s). We define the balancing threshold to be a 
percent imbalance of the given mesh entity order among all of the parts.
The percent is typically between 1\% and 10\%. The stopping criteria can also 
be appended to include problem specific endpoints.
 
  
\section{Gap}

Gap derives from ParMA and thus must define the phases to sepcifically target 
small boundaries and select elements around the small boundaries in order 
to increase the length of these small partition model faces. 

\subsection{Vertex Sides}

We define the sides used in gap to be any two parts that share a patition model 
vertex. We choose this in order to locate single vertex 
junctions between parts. Fixing these single vertex junctions allows gap to 
decrease the number of neighbors by turning the single vertex junction into 
a small boundary that can later be increased. The boundaries are also used 
to count how many vertices a given boundary has. This quantifies the length 
of the boundaries so that Gap can find small partition model faces and increase the length of them.

\subsection{Receiver based targeting}

%General Discussion
The targets of Gap are the parts that share a small boundary. To detect 
these small boundaries the targets must detect themselves rather than the 
senders. Thus, Gap implements a receiver based targeting where parts detect 
that they are receivers and inform the sending parts to send elements to the 
receivers. 

%Description of Algorithm
Algorithm \ref{alg:targets} outlines the method for finding small partition 
model faces 
and determining the sending parts. First each part finds the smallest 
boundary it has. This boundary is targeted as a small partition model face if 
Equation 
\ref{eqn:avgSide} is satisfied where $s$ is the length of the smallest 
boundary, $\bar{s}$ is the average length of all boundaries, and $\alpha$ is 
a percent that defines the upper bound of a small partition model face. 

\begin{equation}
\label{eqn:avgSide}
s < \alpha * \bar{s}
\end{equation}

Each iteration $\alpha$ increases towards 100\% to allow larger boundaries to 
be targeted. Once the small boundaries are found, pairs of parts that share 
these small boundaries send messages to one another to confirm both parts 
agree on the small partition model face. If they disagree then a match is 
not made from one of the parts and the small boundary is ignored for the 
iteration. All of the matches 
that are found become the receivers of this step. The receivers then 
traverse along the small boundary to find the parts that share a vertex 
with the two receiving parts. Figure \ref{fig:boundary} shows an example 
of this. Parts one and two detect the small boundary which then determine 
parts zero and three as the senders. The receivers then send messages to the 
senders to inform them. At that point the senders set each receiver as a 
target and calculates the number of elements to send to the targets defined 
by Equation \ref{eqn:weight}. The weight, $W_i$ is set to 50\% of the amount the small boundary of part $i$, $s_i$, is from the average boundary length. %Is the last sentence overkill/should I explain the 50%/ maybe make it beta and say that we find 50% to have good results 

\begin{equation}
\label{eqn:weight}
W_i = .5*(\bar{s}-s_i)
\end{equation}  

\begin{figure} [!h]
\centering
\captionsetup{justification=centering,margin=1cm}
\includegraphics[width=.6\textwidth]{target_example.png}
\caption{\label{fig:boundary} \textnormal{Four part plate mesh after targeting phase. Part two(red) finds the small bounary with part one(green). Part one labels the parts zero and three (blue) as the senders.}}
\end{figure}

\begin{algorithm}
\caption{Gap Target phase}
\label{alg:targets}
\begin{algorithmic}[1]
\Require {mesh $M$ of dimension $d$, sides $s$ of dimension $0$, weights $w$ of entities of dimension $d$}
\Procedure{targets}{$M$, $s$, $w$}
  \State $targets \leftarrow \emptyset$
  \If {part is in MIS subset AND smallest\_side$<$tolerance}
    \State $other = $ part opposite small partition model face
    \State send message to $other$
  \EndIf
  \If {receives message}
    \State $other = $ message sender
    \State $senders \leftarrow findNeighbors() - self - other$
  \EndIf
  \ForEach {$send \in senders$}
    \State send self and other in message to send
  \EndFor
  \If {receives message}
    \State $targets \leftarrow [parts,calcWeight(parts)]$
    \State $targets \leftarrow [secondpart,calcWeight(second)]$
  \EndIf
\EndProcedure
\end{algorithmic}
\end{algorithm}


\begin{figure} [hb]
\centering
\captionsetup{justification=centering,margin=1cm}
\includegraphics[width=.6\textwidth]{MIS_problem.png}
\caption{\label{fig:recvsend} \textnormal{Possible error if two small boundaries occur next to each other. There are small boundaries between parts nine and eleven, nice and fourteen, and nine and fifteen. Arrows show migration direction for the color of the part finding the small boundary. Conflicts exist between the migrations of the fifteen-fourteen boundary and the fifteen-eleven boundary}}
%REALLY long caption...
\end{figure}

% Maximal Independent Set
One problem that can occur during the target phase is if a part is 
designated as a receiver and a sender. This causes elements to be sent 
to the part and from the part in the same migration which can cause 
disconnected components throughout the boundary between the parts. In Figure 
\ref{fig:recvsend} Gap finds small boundaries between  eleven-nine, 
fouteen-nine, and fifteen-nine.
The arrows in the figure shows the resulting direction of migration. The color 
of an arrow represents the migration caused by the small boundary of that color. 
For example the red arrows are for the fifteen-nine boundary migration. There 
are 
conflicts that occur in the migration along the fourteen-fifteen boundary and 
the eleven-fifteen 
boundary where elements are migrated in both directions.  Our solution to this 
problem is to not allow neighboring parts to send to each other by using a 
maximal independent sets (MIS) of non-neighboring parts. We used Luby's 
randomized parallel MIS algorithm \cite{luby} to separate which parts find the 
small boundaries by making subsets of the parts such that none of the parts 
within a subset are neighbors to each other. Thus each step of the load 
balancing algorithm uses a different subset from the maximal independent sets. 
So the first part of the target phase is each part in the MIS subset 
finds its smallest side. If the side satisfies Equation \ref{eqn:avgSide} 
then the part sends a message to the part on the other side of the 
%Is this better
boundary informing it that the part has a small patition model face. This 
other part then 
determines the senders and the rest of the 
algorithm follows. These changes to the target phase are included in Algorithm 
\ref{alg:targets}. The MIS subsets guarantee that two adjacent parts cannot 
find small boundaries that would cause the sender and receiver conflict. 

\subsection{Selection by Centroid}
After targets have been found each sender must choose which elements close 
to the small partition model face will be migrated to the neighboring parts 
and which part each element will migrate to. To do this Gap's selection phase 
uses geometric centroids to determine which elements to send first and which 
part will receive them. The centroid is found by Equation \ref{eqn:centroid} where $C(p_i$ is the centroid of part $i$, $C(M^d)$ is the centroid of the the ith mesh element of dimension $d$, and $N$ is the number of mesh entities of dimension $d$ in part $i$.
\begin{equation}
\label{eqn:centroid}
C(p_i) = \sum_{M^d \in p_i}C(M^d)/N
\end{equation}
The process begins with the sender part determining a seed 
element that is vertex adjacent to the small side. As there is likely 
multiple elements adjacent to the small side boundary we use the element 
that minimizes Equation \ref{eqn:dist}. 
\begin{equation}
\label{eqn:dist}
D = ||C(p_i)-C(M^d)||
\end{equation}
Once the seed element is selected additional elements are chosen that 
are adjacent by one mesh entity order less than the dimension of the mesh 
(edge in two dimensions and face in three dimensions). Each additional edge 
is added to a priority queue that is sorted by Equation \ref{eqn:dist} from 
highest distance to lowest. This queue is traversed continuously adding more 
elements until the sum of the elements' weight sent is greater than the total 
weight calculated in the targets phase. In Figure \ref{fig:selector} parts A 
and B share a small partition model face and have sender parts C and D. The 
circles on parts A, B, and C mark the part centroids. The element with a star 
is the seed element in selection. The two elements neighboring the seed 
element, marked one and two, are added to the distance queue. Since element 
1 is further from the part centroid, it will be the next element selected. 
The colored elements are the elements in the distance queue after the first 
four elements have been selected.

\begin{figure} [!ht]
\centering
\captionsetup{justification=centering,margin=1cm}
\includegraphics[width=.6\textwidth]{selector_example.png}
\caption{\label{fig:selector} \textnormal{Example of the selection phase. Part C finds the seed element that is furthest from its centroid (circle). Follwing face adjacent elements are chosen by distance from centroid. Shaded elements show the distance queue after 4 elements have been selected. }}
\end{figure}

Since each sender has two target parts, one on each side of the small 
boundary, each element to be migrated must choose one of the target parts to 
migrate to. Each element chooses the part to be migrated to minimize Equation \ref{eqn:dist} where $C(p_i)$ is the centroid of the target parts. 
The closeness of an element to a part is defined as the euclidean distance 
from the elements centroid to the target part's centroid. Thus the element 
is migrated to the receiving part that has the lower of these distances. 
In Figure \ref{fig:selector} the seed element, one and three are sent to 
part A and element two is sent to B.These migration choices lead to the 
small partition model face being extended towards the sender parts. 


\subsection{Stopping Criteria}
% reorder discussion of this at some point.
Gap's goal is to increase the length of small boundaries
Gap does not directly attempt to decrease the imbalance of entities within 
the mesh. In fact, Gap will sacrifice imbalance in order to increase the 
length of partition model boundaries. Therefore the stopping criteria for Gap 
is when it goes above a given level of imbalance. In addition to the imbalance 
criteria Gap also finishes once the minimum boundary length is greater than 
a percentage of the original average boundary length. With both of these 
conditions Gap will complete if the mesh becomes too imbalanced or the 
smallest sides have been widened towards the average. For our tests we 
used 20\% imbalance and 70\% of the initial average side length. We choose 
20\% imbalance because this imbalance is still easily fixed by ParMA entity 
imbalance improvement \cite{parma,zhougraph,zhou2012}.

\chapter{Results}

Our initial testing of Gap has been done on a two dimensional flat plate mesh. 
A 16 part version of the plate mesh partitioned by  the Recursive Inertial 
Bisection (RIB) method \cite{williamsRIB,taylorRIB} is included in Figure 
\ref{fig:plate}. There are single vertex junctions between parts 
zero-one-two-three, 
parts four-five-six-seven, parts one-five-seven-eight, parts 
seven-eight-twelve-forteen, and parts one-three-eight-nine. There are 
also two vertex sides on the thirteen-fourteen boundary, the nine-fifteen 
boundary, and seven-twelve 
boundary. Gap was run on the mesh until the smallest boundary was three after 
nine 
iterations of the balancing algorithm. Figure \ref{fig:plate_gap} shows the 
resulting mesh. Here all of the single vertex junctions have been replaced with 
new part boundaries and increased accordingly. Also several boundaries were 
altered to even out the lengths to at least three. Several larger cases of gap are run and shown in Figures \ref{fig:gap1}, \ref{fig:gap2}, and \ref{fig:gap3}. These are larger 16 part plate meshes with 876 faces, 1273 faces, and 1885 faces respectively.  

\begin{figure} [!hb]
\centering
\captionsetup{justification=centering,margin=1cm}
\includegraphics[width=.6\textwidth]{results_before.png}
\caption{\label{fig:plate} \textnormal{16 part plate mesh with 634 faces before Gap is run. Small boundaries exist between parts thirteen and fourteen, nine and fifteen, etc. Several single vertex junction exist like the one between parts zero, one, two, and three}}
\end{figure}

\begin{figure} [!ht]
\centering
\captionsetup{justification=centering,margin=1cm}
\includegraphics[width=.6\textwidth]{results_after.png}
\caption{\label{fig:plate_gap} \textnormal{The same 16 part plate mesh after Gap has been run. The single vertex junctions have been removed and the minimum side length has been increased to three.}}
\end{figure}


\begin{figure} [!hp]
\centering
\captionsetup{justification=centering,margin=1cm}
\includegraphics[width=.4\textwidth]{before_876.png}
\includegraphics[width=.4\textwidth]{after_876.png}
\caption{\label{fig:gap1} \textnormal{16 part plate mesh with 876 faces before gap (left) and after gap (right)}}
\end{figure}

\begin{figure} [!hp]
\centering
\captionsetup{justification=centering,margin=1cm}
\includegraphics[width=.4\textwidth]{before_1273.png}
\includegraphics[width=.4\textwidth]{after_1273.png}
\caption{\label{fig:gap2} \textnormal{16 part plate mesh with 1273 faces before gap (left) and after gap (right)}}
\end{figure}

\begin{figure} [!hp]
\centering
\captionsetup{justification=centering,margin=1cm}
\includegraphics[width=.4\textwidth]{before_1885.png}
\includegraphics[width=.4\textwidth]{after_1885.png}
\caption{\label{fig:gap3} \textnormal{16 part plate mesh with 1885 faces before gap (left) and after gap (right)}}
\end{figure}


\chapter{Limitations and Future Work} 

%Geometric centroids to topological centroids
Gap is a work in progress and as such has several extensions required to make 
it a more general algorithm. First of all as seen in the Figure \ref{fig:gap3} 
after being balanced by gap, the boundaries are jagged and branch out in a 
nonuniform path which allows longer boundary lengths but a smoother boundary 
and more elliptical shape would be ideal.  These jagged boundaries are a 
result of using geometric centroids to compute distances for migration 
selection. The geometric centroid takes no topological information into account 
which causes the topology to be sacrificed in order to achieve the goal. In 
future testing we would like to try using topological centroids instead of 
geometric centroids. Topological centroids are defined by the distance from the 
boundary by a breadth first search traversal. This gives each vertex a number 
of the shortest path to the boundary. The topological centroid is the vertex 
with the highest number is defined as the topological centroid. Since this 
centroid is defined based on the actual boundary, the distance from the 
centroid is based on element distance rather than element size which leads to 
better topological choices.

%Small Isolated Components and non fixable regions
Due to the nature of Gap's targeting, there are some regions that cannot be 
fixed by Gap. Namely these are areas where there are no sender parts around 
the boundary between two parts. Currently if one of these regions is the 
smallest boundary between the two parts, the parts will continuously try to fix 
those small partition model faces rather than target ones that have sender 
parts incident on the small partition model face. We would like to have Gap 
ignore these un-fixable boundaries in order to target small partition model 
faces that can be fixed by the algorithm. In Figure \ref{fig:ocean} there are 
several small isolated components that have been highlighted that cause a 
small boundary to be detected with the main part. These sides cannot be 
increased because there are no parts next to the boundary to migrate elements 
to the red or orange parts. Thus Gap stagnates in this portion of the mesh 
for any small partition model faces that are bigger than the sum of the 
isolated components boundaries 

\begin{figure} [!ht]
\centering
\captionsetup{justification=centering,margin=1cm}
\includegraphics[width=.65\textwidth]{Ocean_Isolated.png}
\caption{\label{fig:ocean} \textnormal{Several isolated components that cause the two parts to have small boundaries that cannot be fixed by Gap since there are no parts next to the boundary.}}
\end{figure}


\chapter{Closing Remarks}

We presented a method to improve the shape of a mesh in order to reduce 
communication costs across different processes. Out implementation worked off of 
a parallel dynamic partitioning framework, ParMA, to target small boundaries 
and migrate elements at the end of the boundaries based on distances from 
centroids to increase the length sides. The key to our method is improving 
shape at the cost of small imbalance increases which can be quickly recovered 
with ParMA diffusive improvement procedures without losing the shape 
improvements. Our initial testing shows for a simple two dimensional mesh that 
Gap successfully lengthens the small boundaries and removes single vertex junctions. 
Since the plate mesh is 
simplistic there were several extra considerations that must be implemented 
to create a method that works in general cases of both two dimensional and 
three dimensional meshes. 

\endgroup

\specialhead{LITERATURE CITED}

%\newpage
\bibliographystyle{plain}
\bibliography{references.bib}

\end{document}
