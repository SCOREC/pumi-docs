%%%%%%%%%%%%%%%%%%%%%%%%%%%%%%%
\section{System-wide API's}
%%%%%%%%%%%%%%%%%%%%%%%%%%%%%%%

This section describes API functions and enumeration types which are not bounded with a specific geometric model or mesh data.

\begin{verbatim}
void pumi_start()		
\end{verbatim}
\vspace{-.5cm}\hspace{1cm}
	Initialize parallel services pertinent to PUMI.

\begin{verbatim}	
void pumi_finalize(bool /* in */ do_mpi_finalize=false)		
\end{verbatim}\vspace{-.5cm}\hspace{1cm}
	Finalize parallel services and clean the memory. If the input parameter \emph{do}$\_$\emph{mpi}$\_$\emph{finalize} is true, MPI finalization is performed as well. \emph{do}$\_$\emph{mpi}$\_$\emph{finalize} is optional (default: \emph{false}).

\begin{verbatim}
int pumi_size()
\end{verbatim}\vspace{-.5cm}\hspace{1cm}
	Return the number of processes.

\begin{verbatim}
int pumi_rank()
\end{verbatim}	
\vspace{-.5cm}\hspace{1cm}
	Return the MPI rank in communicator. Rank starts from 0. 

\begin{verbatim}
void pumi_sync()
\end{verbatim}\vspace{-.5cm}\hspace{1cm}
	Synchronize all processes in communicator

\begin{verbatim}
void pumi_printSys()
\end{verbatim}\vspace{-.5cm}\hspace{1cm}
	Print system information such as host name, processor, operating system, etc.\\
\emph{(Example)} Linux node10.borg.scorec.rpi.edu 2.6.9-89.ELsmp SMP Mon Jun 22 12:31:33 EDT 2009 x86$\_$64.

\begin{verbatim}
double pumi_getMem()
\end{verbatim}\vspace{-.5cm}\hspace{1cm}
	Return the heap memory increase (MB) on local process since \emph{pumi}$\_$\emph{start()}.

\begin{verbatim}
double pumi_getTime()
\end{verbatim}\vspace{-.5cm}\hspace{1cm}
	Return the current time in second.


\begin{verbatim}
void pumi_printTimeMem(
        const char* /* in */ msg, 
        double /* in */ time, 
        double /* in */ memory)
\end{verbatim}\vspace{-.5cm}\hspace{1cm}
        Display $``$\emph{msg}: \emph{time} (sec) and \emph{memory} (MB)$"$.

