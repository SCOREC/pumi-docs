\section{Introduction}
\addcontentsline{toc}{section}{Introduction}

\subsection{Background and Motivation}
An efficient distributed mesh data structure is needed to support
parallel adaptive analysis since it strongly influences
the overall performance of adaptive mesh-based simulations. In addition to
the general mesh-based operations~\cite{beall97}, such as mesh entity creation/deletion,
adjacency and geometric classification, iterators, arbitrary 
attachable data to mesh entities, etc., the distributed
mesh data structure must support $(i)$ efficient communication between
 entities duplicated over multiple processors, $(ii)$ migration of mesh entities
 between processors, and $(iii)$ dynamic load balancing. 

Issues associated with supporting parallel adaptive
analysis on unstructured meshes include dynamic mesh
load balancing techniques~\cite{deCougny-etal95, Diekmann-etal00, Teresco-etal00, Walshaw-Cross00}, and data structure and algorithms for parallel 
mesh adaptation~\cite{Ozturan-etal94, deCougny-Shephard99, libmesh, Oliker-etal00, Park-Kwon05, Remacle-etal02, Selwood-Berzins99, seolthesis, fmdb06}. 

The PUMI is a unstructured, distributed mesh data management system that is capable of a parallel mesh infrastructure capable of handling general non-manifold \cite{Mantyla88, weiler88} models and
effectively supporting automated adaptive analysis \cite{pumi16}. 

This document describes an overview of the PUMI design, its interface functions and example codes.

\subsection{Organization}

Chapter 2 introduces the data sets involved with geometry-based analysis and the role of a topological mesh representation. Chapters 3-5 introduce the PUMI libraries which include parallel control, geometric model, distributed mesh data structure in accordance with a partition model, and mesh partitioning control. Chapters 6-11 present the PUMI API specifications. Chapter 12 presents example programs. Chapter 13 provides compilation and installation instructions.

\subsection{Nomanclature}
	
\begin{tabular}{lp{14cm}}
$V$ & the model, $V$ $\in$ \{$G$, $P$, $M$\} where $G$ signifies the
geometric model, $P$ signifies the partition model, and
 $M$ signifies the mesh model.\\
\{$V\{V^d$\}\} & a set of topological entities of dimension $d$ in model $V$. 
(e.g., \{$M\{M^2$\}\} is the set of all the faces in the mesh.)\\
$V^d_i$ & the $i^{th}$ entity of dimension\footnote{In this document, entity dimension and entity type are interchangeable.} $d$ in model $V$. $d=0$ for a vertex, $d=1$ for an edge, $d=2$ for a face, and
$d=3$ for a region.\\
\{$\partial$($V^d_i$)\} & set of entities on the boundary of $V^d_i$. \\ 
\{$V^d_i$\{$V^q$\}\} &a set of entities of dimension $q$ in model $V$ that are
adjacent to $V^d_i$. (e.g., \{$M^1_3$\{$M^3$\}\} are the mesh regions adjacent to mesh edge $M^1_3$.) \\

$V^d_i$\{$V^q$\}$_j$ & the $j^{th}$ entity in the set of entities of
dimension $q$ in model $V$ that are adjacent to $V^d_i$. 
(e.g., $M^3_1$\{$M^1$\}$_2$ is the $2^{nd}$ edge adjacent to mesh region $M^3_1$.) \\
$U^{d_i}_i$ $\sqsubset$ $V^{d_j}_j$ & classification indicating the unique association
of entity $U^{d_i}_i$ with entity $V^{d_j}_j$, $d_i \le d_j$, where $U$, $V$ $\in$
\{$G$, $P$, $M$\} and $U$ is lower than $V$ in terms of a hierarchy of domain decomposition.\\
\RP$[$$M^d_i$$]$ & set of part id(s) where entity $M^d_i$ exists.\\

\end{tabular}
