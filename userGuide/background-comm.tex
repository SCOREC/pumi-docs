\section{Parallel Control and Communication}  

The PCU (Parallel Control Utility) is a library for parallel computation based on
MPI with additional support for hybrid MPI/thread environments. PCU is included in PUMI to support needed parallel controls and communication for operations with distributed meshes and model.

The PCU provides three things to users:

\begin{enumerate}
\item A phased message passing API
\item A thread management API
\item Support for phased message passing between threads instead of processes
\end{enumerate}

Phased message passing is similar to Bulk Synchronous Parallel,
but is implemented more efficiently.
All messages are exchanged in a phase, which is a collective operation involving
all threads in the parallel program.
During a phase, the following events happen in sequence:

\begin{enumerate}
\item All threads send non-blocking messages to other threads
\item All threads receive all messages sent to them during this phase
\end{enumerate}

PCU provides termination detection, which is the ability to detect when all messages
have been received without prior knowledge of which threads are sending to which.

To write hybrid MPI/thread programs, PCU provides a function that creates threads
within an MPI process, similar to the way mpirun creates multiple processes.
PCU assigns ranks to these threads and has them each run the same function, with
thread-specific input arguments to the function.

Once a program has created threads using PCU, it can call the phased message passing
API from within threads, which will behave as if each thread were an MPI process.
Threads have unique ranks and can send messages to one another, regardless of which
process they are in.


