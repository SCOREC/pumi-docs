\section{Interface Structure}

The PUMI API's are provided in the file \texttt{``pumi.h''}. Please be noted that in the current PUMI release, an entity set and multiple parts per process are not supported. Herein, with a single part per process, a mesh instance and a part handle are interchangeable.

\subsection{API naming convention}

PUMI API function name consists of two or three words connected with `$\_$'.

\begin{itemize}
\item the first word is ``pumi"
\item the second word is an operation target. If the operation target is system-wide, the operation target is ommited. For instance, the function that initializes the PUMI service is \emph{pumi}$\_$\emph{start}.
\item the third word is the operation description starting with a verb. For example, the function \emph{pumi}$\_$\emph{mesh}$\_$\emph{getNumEnt} returns the number of mesh entities with specific type. The function \emph{pumi}$\_$\emph{gent}$\_$\emph{getID} returns the global ID of geometric model entity.
\end{itemize}

The following are operation targets used in the second word.

\begin{itemize}
\item \textit{geom}: the api is performed on a geometric model
\item \textit{gent}: the api is performed on a geometric model entity
\item \textit{mesh}: the api is performed on a local mesh or global mesh
\item \textit{ghost}: the api performs ghosting functions
\item \textit{ment}: the api is performed on a mesh entity
\item \textit{tag}: the api is performed on a tag handle
\item \textit{field}: the api is performed on a field object
\item \textit{shape}: the api is performed on a field shape object
\item \textit{node}: the api is performed on a field node
\item \textit{numbering}: the api is performed on a numbering object
\end{itemize}

\subsection{Abbreviation}

Abbreviations may be used in API naming. See \textit{http://scorec.rpi.edu/wiki/Abbriviations} for more information.
 
\subsection{Data Types and Classes}

For a geometry, partition and mesh model, the term \emph{instance} is used to indicate the model data existing on each process. For example, a mesh instance on process \emph{i} means a pointer to a mesh data structure on process \emph{i}, in which all mesh entities on process \emph{i} are contained and from which they are accessible. For all other data such as entity and entity set, the term \emph{handle} is used to indicate the pointer to the data. For example, a mesh entity handle means a pointer to the mesh entity data. The predefined data type has a prefix \emph{p} to indicate the pointer data type.

The following are predefined data types used in the interface function parameters.

\begin{tabular}{lp{14cm}}	
	pGeom &		geometric model instance \\
	pGeomEnt &	geometric entity handle \\
	&\\
	pMesh &		mesh instance\\
        pMeshEnt &	mesh entity handle (i.e., vertex, edge, face, region) \\
        &\\
        pOwnership &    an ownership handle to allow user-defined ownership rule for part boundary entities\\
        Vector3 & 	an array of three doubles to hold the coordinate information of node\\
        Adjacent & 	an array of entities to hold the adjacency\\
        EntityVector  &    std::vector of mesh entity handle (pMeshEnt) \\
        Copies &          std::map of part ID (int) and mesh entity handle (pMeshEnt) to hold remote or ghost copies \\
        Parts  &          std::set of part ID (int) \\
	&\\
        pTag &          tag handle for geometric model\\
        pMeshTag &      tag handle for mesh\\
        &\\
	pGeomIter &		iterator traversing over global geometric model entities \\
        pMeshIter &          iterator traversing over local mesh entities\\
        pCopyIter &       iterator traversing \texttt{Copies} \\
        &\\
	pShape &	     shape function handle to define how the field nodes are distributed \\
        pField &             field handle\\
        pNumbering & numbering handle to assign local numbering to multiple degree of freedoms in field nodes \\
        pGlobalNumbering &   global numbering handle to assign global numbering to multiple degree of freedoms in field nodes\\
\end{tabular}	

The following classes are defined to support mesh re-distribution and ghosting.

\begin{tabular}{lp{14cm}}	
	Migration &	a class to define migration plan \\
	Distribution &	a class to define distribution plan \\
	Ghosting &	a class to define ghosting plan \\
\end{tabular}	

\subsection{Enumeration Types}

The enumeration type for tag data type is:

\begin{verbatim}
    enum PUMI_TagType {
        PUMI_DBL       = 0 /* double */,
        PUMI_INT,      /* 1 - integer */
        PUMI_LONG,     /* 2 - long integer */
        PUMI_ENT,      /* 3 - entity handle. Only for geometric model*/
        PUMI_SET,      /* 4 - set handle. NOT SUPPORTED */
        PUMI_PTR,      /* 5 - opaque pointer. Only for geometric model*/
        PUMI_STR,      /* 6 - string. NOT SUPPORTED */
        PUMI_BYTE      /* 7 - 1 byte character. Only for geometric model */
    }
\end{verbatim}\vspace{-1cm}\hspace{1cm}

For geometric model, the supported tag types are \emph{PUMI$\_$DBL}, \emph{PUMI$\_$INT}, \emph{PUMI$\_$LONG}, \emph{PUMI$\_$ENT}, \emph{PUMI$\_$PTR}, and \emph{PUMI$\_$BYTE}. For mesh, the supported tag types are \emph{PUMI$\_$DBL}, \emph{PUMI$\_$INT}, and \emph{PUMI$\_$LONG}.

The enumeration type for entity topology is:

\begin{verbatim}
enum PUMI_EntTopology {
  PUMI_VERTEX, 		// 0 
  PUMI_EDGE,   		// 1 
  PUMI_TRIANGLE, 	// 2 
  PUMI_QUAD,		// 3
  PUMI_TET,  		// 4
  PUMI_HEX,  		// 5 
  PUMI_PRISM, 		// 6
  PUMI_PYRAMID 		// 7
};
\end{verbatim}\vspace{-1cm}\hspace{1cm}

The enumeration type for field data type is:
\begin{verbatim}
enum PUMI_FieldType {
  PUMI_SCALAR, 		//  a single scalar value
  PUMI_VECTOR, 		// a 3D vector
  PUMI_MATRIX, 		// a 3x3 matrix 
  PUMI_PACKED 		// a user-defined set of components
};
\end{verbatim}\vspace{-1cm}\hspace{1cm}

\subsection{Error Codes}
If the API function returns an error code and the function succeeds, it returns 0,  otherwise, it returns a positive integer representing a type of error. The error codes are defined in \emph{pumi}$\_$\emph{errorcode.h}. 

\begin{verbatim}
 enum PUMI_ErrCode {
  PUMI_SUCCESS = 0, // no error
  PUMI_MESH_ALREADY_LOADED,
  PUMI_FILE_NOT_FOUND,
  PUMI_FILE_WRITE_ERROR,
  PUMI_NULL_ARRAY,
  PUMI_BAD_ARRAY_SIZE,
  PUMI_BAD_ARRAY_DIMENSION,
  PUMI_INVALID_ENTITY_HANDLE,
  PUMI_INVALID_ENTITY_COUNT,
  PUMI_INVALID_ENTITY_TYPE,
  PUMI_INVALID_ENTITY_TOPO,
  PUMI_BAD_TYPE_AND_TOPO,
  PUMI_ENTITY_CREATION_ERROR,
  PUMI_INVALID_TAG_HANDLE,
  PUMI_TAG_NOT_FOUND,
  PUMI_TAG_ALREADY_EXISTS,
  PUMI_TAG_IN_USE, // try to delete a tag that is in use
  PUMI_INVALID_SET_HANDLE,
  PUMI_INVALID_ITERATOR,
  PUMI_INVALID_ARGUMENT,
  PUMI_MEMORY_ALLOCATION_FAILED,
  PUMI_INVALID_MESH_INSTANCE,
  PUMI_INVALID_GEOM_MODEL,
  PUMI_INVALID_GEOM_CLAS,
  PUMI_INVALID_PTN_CLAS,
  PUMI_INVALID_REMOTE,
  PUMI_INVALID_MATCH,
  PUMI_INVALID_PART_HANDLE, 
  PUMI_INVALID_PART_ID,
  PUMI_INVALID_SET_TYPE,
  PUMI_INVALID_TAG_TYPE,
  PUMI_ENTITY_NOT_FOUND,
  PUMI_ENTITY_ALREADY_EXISTS,
  PUMI_REMOTE_NOT_FOUND,
  PUMI_GHOST_NOT_FOUND,
  PUMI_CB_ERROR,
  PUMI_NOT_SUPPORTED,
  PUMI_FAILURE, 
}
\end{verbatim}
