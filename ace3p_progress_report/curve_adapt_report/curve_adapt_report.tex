%%%\documentclass[%
%%%%reprint,
%%%%superscriptaddress,
%%%%groupedaddress,
%%%%unsortedaddress,
%%%%runinaddress,
%%%%frontmatterverbose, 
%%%preprint,
%%%%showpacs,preprintnumbers,
%%%%nofootinbib,
%%%%nobibnotes,
%%%%bibnotes,
%%% amsmath,amssymb,
%%%%aps,
%%%%pra,
%%% prb,
%%%%rmp,
%%%%prstab,
%%%%prstper,
%%%floatfix,
%%%%nolongbibliography
%%%]{revtex4-1}

\documentclass[review,authoryear,12pt]{elsarticle_summary_report}
\usepackage[top=1.0in, bottom=1.0in, left=1in, right=1in]{geometry}


%%%%%%%%%%%%%%%%%%%%
%\usepackage{amsfonts}
%\usepackage{amssymb}
\usepackage{MnSymbol}
\usepackage{graphicx}
\usepackage{psfrag}
\usepackage{amsmath}
\usepackage[usenames]{color}
\usepackage{leftidx}
\usepackage[small]{subfigure}
\usepackage{stmaryrd}
\usepackage{amsthm}
\usepackage{multirow}
\usepackage[table]{xcolor}
\usepackage{natbib}
\usepackage{nomencl}
\usepackage{setspace}
\usepackage{dcolumn}% Align table columns on decimal point
\usepackage{bm}% bold math
\usepackage{pdflscape}
% \usepackage{showkeys}
%%%%%%%%%%%%%%%%%%%%

\usepackage{hyperref}
\hypersetup{
    colorlinks=true,
    linkcolor=blue,
    filecolor=magenta,      
    urlcolor=cyan,
}

% \usepackage[active,tightpage]{preview}
% \PreviewSnarfEnvironment[{[]}]{figure}

\makenomenclature

\graphicspath{ {./Figures/cav_17_results/}
               {./Figures/pillbox_coarse_uniform_results/}   
             }

%\renewenvironment{equation}[0]{equation}{equation}

%%%%%%%%%%%%%%%%%%%% My Commands
\newcommand*\diff{\mathop{\mathrm{d}\hspace*{-3pt}}}
\newcommand{\tone}{{ {(1)}}}				%%upper par (1)
\newcommand{\ttwo}{{ {(2)}}}				%%upper par (2)
\newcommand{\ptwo}{{\vphantom{ {(2)}}}}    %%V-Phantom upper par (2)
\newcommand{\ti}[1]{{ {(#1)}}}				%%upper par (i)
\newcommand{\si}[1]{{ {[#1]}}}				%%upper par [i]
\newcommand{\VB}[1]{{ {\text{#1}\color{red}\veebar\color{black}}  }}				


\newcommand{\sym}{\boldsymbol{\sigma}}		%%Symmetric Polarization
\newcommand{\asym}{\boldsymbol{\tau}}		%%AntiSymmeteric Polarization

%%%%%%%%%%%%%%%%%%%%%%%%%%%%%%%%
\newcommand{\bs}[1]{\boldsymbol{#1}}						%%bold sym
\newcommand{\abs}[1]{\left | {#1} \right |}						%%absolute value
\newcommand{\pr}[1]{\left( #1 \right)}							%%paranthesis
\newcommand{\br}[1]{\left[ #1 \right]}							%%bracet
\newcommand{\cbr}[1]{\left\lbrace #1 \right\rbrace}				%%c bracet
\newcommand{\avg}[1]{\left \llangle #1 \right \rrangle}				%%average
\newcommand{\avgn}[1]{\left\langle #1 \right\rangle}				%%average
\newcommand{\jump}[1]{\left \llbracket #1 \right \rrbracket }	%%jump
\newcommand{\integral}[3]{\int _{#1}  #2 \;\; d #3}				%%integral

\newcommand{\Pchoose}[2]{\frac{#1!}{#2!}}

\newcommand{\mref}[2]{(#1)$_{\text{#2}}$}						%%equation reference

\newcommand{\mud}{m}		%%superscrip mg
\newcommand{\mg}{{\textit{\scriptsize mg}}}		%%superscrip mg
\newcommand{\mxw}{{\textit{\scriptsize M}}}		%%superscrip M
\newcommand{\ex}{{\textit{\scriptsize ex}}}		%%superscrip ex
\newcommand{\el}{{\textit{\scriptsize el}}}		%%superscrip el
\newcommand{\es}{{\textit{\scriptsize es}}}		%%superscrip es
\newcommand{\me}{{\textit{\scriptsize me}}}		%%superscrip el
\newcommand{ \elt}{{\textit{\tiny el}}}			%%subscrip   el
\newcommand{\dist}[1]{\!_{\mathcal{D}_{#1}}}
\newcommand{\incs}[1]{_{\mathcal{I}_{#1}}}

\newcommand{\FTF}{\boldsymbol{F} ^ \trans \boldsymbol{F}}	%%F transpose F
\newcommand{\FT}{\boldsymbol{F} ^ \trans}					%%F transpose
\newcommand{\FTD}{\boldsymbol{F} ^ \trans \mathbf{d}}		%%F transpos d
\newcommand{\trans}{{\textit{\tiny{T}}}}					%%transpose
%%%%%%%%%%%%%%%%%%%%
\newcommand{\FBar}{\bar{\boldsymbol{F}}}
\newcommand{\UBar}{\bar{\boldsymbol{U}}}
\newcommand{\QBar}{\bar{\boldsymbol{Q}}}
\newcommand{\LamBar}{\bar{\boldsymbol{\Lambda}}}
\newcommand{\RBar}{\bar{\boldsymbol{R}}}
\newcommand{\CBar}{\bar{\boldsymbol{C}}}
\newcommand{\DBar}{\bar{\mathbf{D}}}
\newcommand{\EBar}{\bar{\mathbf{E}}}
\newcommand{\dBar}{\bar{\mathbf{d}}}
\newcommand{\eBar}{\bar{\mathbf{e}}}
\newcommand{\JBar}{\bar{J}}
\newcommand{\epsTilde}{\tilde{\boldsymbol{\varepsilon}}}
\newcommand{\EpsTilde}{\tilde{\boldsymbol{\mathcal{E}}}}
\newcommand{\XTilde}{\tilde{\boldsymbol{\mathcal{X}}}}
\newcommand{\KTilde}{\tilde{\boldsymbol{\mathcal{K}}}}
\newcommand{\I}{\boldsymbol{I}}
\newcommand{\TBar}{\bar{\boldsymbol{T}}}

%%%%%%%%%%%%%%%%%%%%
%\numberwithin{equation}{section}  	%%Equation Numbering
%%%%%%%%%%%%%%%%%%%%
\begin{document}

\title{TITLE}% Force line breaks with \\
%\thanks{}%

\author[]{Morteza H. Siboni \\
Gerrett Diamond \\
Cameron Smith}
%\ead{email address}

%\author[inst1]{Corresponding Author \corref{cor1}}
%\cortext[cor1]{Corresponding author}
%\ead{ca@email.host.edu}
%\address[inst1]{Department of Mechanical Engineering and Applied Mechanics, University of Pennsylvania, \\ Philadelphia, PA 19104-6315, USA}


\date{\today}



% \begin{abstract}
% \end{abstract}

% \begin{keyword}
% \end{keyword}

\maketitle


% \begin{spacing}{0.5}
% \printnomenclature
% \end{spacing}


\section{Introduction}

\section{Curved Mesh Adaptation}

\begin{landscape}
\begin{figure}[ph!]
\centering
\subfigure[]{\includegraphics[width=0.55\textwidth]{al_0_ar_0p0125_126044_elems.png}}
\hspace*{50pt}
\subfigure[]{\includegraphics[width=0.55\textwidth]{al_0_ar_0p0125_126044_elems_size_field.png}}
\\
\subfigure[]{\includegraphics[width=0.55\textwidth]{al_3_ar_0p0125_386896_elems.png}}
\hspace*{50pt}
\subfigure[]{\includegraphics[width=0.55\textwidth]{al_3_ar_0p0125_386896_elems_e_field.png}}
\caption{This Figure shows the results for the CAV17 model. (a) shows the initial mesh [$\sim126\text{K}$ elements], (b) shows the initial size-field, (c) shows the adapted mesh after 3 adaptation steps [$\sim380\text{K}$ elements], and (d) shows the electric field for the final adapted mesh.}
\end{figure}
\end{landscape}


\begin{landscape}
\begin{figure}[ph!]
\centering
\subfigure[]{\includegraphics[width=0.55\textwidth]{al_0_ar_0p0125_3721_elems.png}}
\hspace*{50pt}
\subfigure[]{\includegraphics[width=0.55\textwidth]{al_0_ar_0p0125_3721_elems_size_field.png}}
\\
\subfigure[]{\includegraphics[width=0.55\textwidth]{al_3_ar_0p0125_14221_elems.png}}
\hspace*{50pt}
\subfigure[]{\includegraphics[width=0.55\textwidth]{al_3_ar_0p0125_14221_elems_e_field.png}}
\caption{This Figure shows the results for the PILLBOX model. (a) shows the initial mesh [$\sim3.7\text{K}$ elements], (b) shows the initial size-field, (c) shows the adapted mesh after 3 adaptation steps [$\sim14\text{K}$ elements], and (d) shows the electric field for the final adapted mesh.}
\end{figure}
\end{landscape}



\section{Load Balance}

\section{Moving Towards Higher-Order Geometries}

% \section*{References}
% \bibliographystyle{elsarticle-harv_noURL}
% \bibliography{Ref} 


\end{document}
